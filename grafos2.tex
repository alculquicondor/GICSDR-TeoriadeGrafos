\section{Conexidad}

\begin{definicion}
\label{"grafo conexo"}
Un grafo es conexo si $\forall x,y\in V_G$ existe al menos un camino de $x$ a $y$.
\end{definicion}

Decimos que $x$ está ligado a $y$ si existe un camino de $x$ a $y$.

\begin{teorema}
\label{teoconex}
En un grafo $G$ conexo, se cumple:
\begin{enumerate}
\item $m(G)\ge n(G)-1$
\item $\exists v \in V_G$ tal que $G-v$ es conexo.
\end{enumerate}
\end{teorema}

\begin{proof}
\begin{enumerate}
\item El grafo conexo más simple es un camino de $n$ vértices, el cual tiene $n-1$ aristas.
\item Podemos tomar cualquier vértice $v$ que sea extremo en un camino maximal del grafo.
Por definición, todas sus aristas apuntan a otros vértices del camino, los cuales están unidos por el mismo camino, por tanto, eliminando $v$ no afectamos la conexidad de éstos.
\end{enumerate}
\end{proof}

\begin{teorema}
Sea $G$ un grafo $(U,W)$-bipartido, tal que $\delta(G)>k/2$ entonces $G$ es conexo. $k=max(|U|,|W|)$
\end{teorema}

%\begin{proof}
%Demostración gráfica. Después lo hago
%\end{proof}

\begin{teorema}
Sean $P^*$ y $Q^*$ dos caminos máximos en un grafo conexto $G$, entonces $E_{P^*} \cap E_{Q^*}\neq \emptyset$
\end{teorema}

%\begin{proof}
%Demostración gráfica por contradicción
%\end{proof}

\begin{definicion}
Un subgrafo conexo $H$ de un grafo $G$ es maximal (con relación a la conexidad) si no existe un subgrafo conexo $H'$ tal que $H \subset H'\subseteq G$
\end{definicion}

\begin{definicion}
Un componente de un grafo $G$ es cualquier subgrafo maximal de $G$.
\end{definicion}

\notacion $C(G)$ es el número de componentes de $G$.

\begin{teorema}
En todo grafo $G$, se cumple que $m(G)\ge n(G)-C(G)$.
\end{teorema}

\begin{proof}
Si $G$ es conexo, obtenemos $m(G)\ge n(G)-1$, que es el teorema \ref{teoconex}.

Si $G$ no es conexo, aplicamos el teorema \ref{teoconex} en cada componente de $G$:
$$m(G_i)\ge n(G_i)-1$$
Aplicando sumatoria a ambos lados:
$$\sum_i m(G_i)\ge\sum_i (n(G_i)-1)$$
$$\sum_i m(G_i)\ge\sum_i n(G_i)-\sum_i1$$
$$m(G)\ge n(G)-C(G)$$
\end{proof}

\subsection*{Ejercicios}
Diseñe los siguientes algoritmos:
\begin{enumerate}
\item {\bf Algoritmo:} Entrada: $G$ y $v\in V_G$

Salida: Componente de $G$ que contiene a $v$
\item {\bf Algoritmo:} Entrada: $G$

Salida: $C(G)$
\end{enumerate}



\section{Bicolorabilidad}

\begin{definicion}
Un grafo $G$ es bicolorable si existe una partición $(U,W)$ tal que toda arista de G tiene una punta en $U$ y la otra punta en $W$.
\end{definicion}
También se dice que el grafo es potencialmente bipartible.

\begin{teorema}
Un grafo es bicolorable si y sólo si no hay un circuito impar en él.
\end{teorema}

\begin{proof}
Probaremos primero que si existe un circuito impar en el grafo, entonces éste no es bicolorable.
Asignamos a cualquier vértice del circuito un color 1 y recorriendo el mismo en sentido antihorario intercalamos con el color 2. El último vértice en pintarse será de color 1 y, como por definición de circuito es adyancente al primero, el grafo no es bicolorable.

Para probar que si un grafo es bicolorable no existe un circuito impar en él, usaremos inducción en el número de aristas.
\end{proof}
